%----------------------------------------------------------------------------------------
%	PACKAGES AND OTHER DOCUMENT CONFIGURATIONS
%----------------------------------------------------------------------------------------

\documentclass[12pt]{article}

\usepackage[utf8]{inputenc} % Required for inputting international characters
\usepackage[T1]{fontenc} % Output font encoding for international characters
\usepackage[french]{babel}
\usepackage{url}
\usepackage{mathpazo} % Palatino font
\usepackage{graphicx} % Import images in document
\graphicspath{ {./images/} }


\begin{document}

%----------------------------------------------------------------------------------------
%	TITLE PAGE
%----------------------------------------------------------------------------------------

\begin{titlepage} % Suppresses displaying the page number on the title page and the subsequent page counts as page 1
	\newcommand{\HRule}{\rule{\linewidth}{0.5mm}} % Defines a new command for horizontal lines, change thickness here
	
	\center % Centre everything on the page
	
	%------------------------------------------------
	%	Headings
	%------------------------------------------------
	
	\textsc{\LARGE Université de Namur}\\[1.5cm] % Main heading such as the name of your university/college
	
	\textsc{\Large Initiation à la démarche scientifique}\\[0.5cm] % Major heading such as course name
	
	\textsc{\large Fiche de lecture\\[0.5cm]} % Minor heading such as course title
	
	%------------------------------------------------
	%	Title
	%------------------------------------------------
	
	\HRule\\[0.4cm]
	
	{\huge\bfseries Complex Event Processing for Internet of Things }\\[0.4cm] % Title of your document
	
	\HRule\\[1.5cm]
	
	%------------------------------------------------
	%	Author(s)
	%------------------------------------------------
	
	\begin{minipage}{0.4\textwidth}
		\begin{flushleft}
			\textit{Auteur}\\
			
			\large Kenny \textsc{Warszawski}
		\end{flushleft}

	\end{minipage}
	~
	\begin{minipage}{0.5\textwidth}
		\begin{flushright}
			\large
			\textit{Professeurs}\\
			M. Amrani \textsc{Moussa}\\
			M. Schobbens \textsc{Pierre-Yves}
		\end{flushright}
	\end{minipage}
	
	% If you don't want a supervisor, uncomment the two lines below and comment the code above
	%{\large\textit{Author}}\\
	%John \textsc{Smith} % Your name
	
	%------------------------------------------------
	%	Date
	%------------------------------------------------
	
	\vfill\vfill\vfill % Position the date 3/4 down the remaining page
	
	{\large\today} % Date, change the \today to a set date if you want to be precise
	
	%------------------------------------------------
	%	Logo
	%------------------------------------------------
	
	\vfill\vfill
	\includegraphics[width=0.2\textwidth]{placeholder.png}\\[1cm] % Include a department/university logo - this will require the graphicx package
	 
	%----------------------------------------------------------------------------------------
	
	\vfill % Push the date up 1/4 of the remaining page
	
\end{titlepage}
%----------------------------------------------------------------------------------------
\newpage
%----------------------------------------------------------------------------------------

\begin{abstract}
\normalsize
(Attention !!! Ce résumé est dédié uniquement à cette première version.)\\  \\
Le premier objectif de cette fiche de lecture est de pouvoir définir et expliquer les différents termes importants à la compréhension de ce document. Le second, est d'analyser certains articles scientifiques concernant la thématique. Enfin, un objectif est de définir des limites au sujet choisi. En effet, le traitement d'évènements complexes porté à l'internet des objets est un sujet très vaste. Par conséquent, il est nécessaire d'y fixer des bornes.

\end{abstract}

\newpage

\section{Contexte}

ATTENTION Je dois rajouter les références vers les articles de manière correcte et refaire la structure de la citation.
\\

\textbf{Complex Event Processing} : Selon l'article "Formalizing Complex Event Processing Systems in Maude": "Le traitement des événements complexes (CEP) est une technologie de pointe permettant d'analyser et de mettre en corrélation des flux d'informations sur les événements se produisant dans un système et d'en tirer des conclusions. Le CEP permet de définir des événements complexes basés sur les événements produits par les sources entrantes, d'identifier des situations complexes et significatives et d'y répondre le plus rapidement possible."\\

\textbf{Internet des objets (IoT)} : Selon l'International Telecommunication Union, il s'agit de: "L'infrastructure mondiale pour la société de l'information, qui permet de disposer de services évolués en interconnectant des objets (physiques ou virtuels) grâce aux technologies de l'information et de la communication interopérables existantes ou en évolution." \\

\textbf{Objet} : Selon l'International Telecommunication Union : "Dans l'Internet des objets, objet du monde physique (objet physique) ou du monde de l'information (objet virtuel), pouvant être identifié et intégré dans des réseaux de communication."
\\ \\
Sources : \\
\url{https://ieeexplore.ieee.org/document/8352137} \\
\url{https://www.itu.int/ITU-T/recommendations/rec.aspx?rec=11559&lang=fr}


\section{Analyse}

Cette section est dédiée à l'analyse de différents articles qui correspondent au theme de la fiche de lecture.


\subsection{Article 1}

Article: \textit{\url{https://ieeexplore.ieee.org/document/8288619}}
\\\\



\subsection{Article 2}

Article: \textit{\url{https://www.sciencedirect.com/science/article/pii/S0898122113004677}}

\subsection{Article 3}

Article: \textit{\url{https://journals.sagepub.com/doi/pdf/10.1155/2014/159052}}

\section{Limites}

\end{document}
