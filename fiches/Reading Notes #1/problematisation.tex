\documentclass{article}

\input{structure.tex}

\bibliography{biblio}

\title{IHDCB339: Problematisation}

\author{Kenny Warszawski\\ \texttt{kenny.warszawski@student.unamur.be}}

\date{University of Namur --- \today}

\begin{document}

\maketitle 

\section{Problematisation}

Nowadays, Internet of Things is omnipresent and the amount of connected devices are increasing day by day. Those devices are used in a lot of sectors and they can be different from one another. That's why they need to integrate on a system in an heterogeneous way. Some sectors need to have extremely fast responses. That's why IoT can be coupled to CEP engines to have extremely fast data treatments. ThingML with CEP extension helps the implementation of such complex system by adding an abstraction on the programming language used for deployment. That way, a model described in ThingML has the same behaviour even if it is deployed in Java or C/C++. This technology revealed to have good performance even if the physical resources are limited. 
\\ \\
ThingML is a lightweight and open source Event Pattern Language for IoT and Complex Event Processing but some questions are interesting to ask. How this solution can be deployed on a real complex system ? How all those devices can communicate with each other ? Which protocol should be used ? Which deployment strategy can fit with an heterogeneous and continuous delivery industry ? All those questions are interesting to answer. Smart Cities can be an interesting use case for the following analysis.

\end{document}
