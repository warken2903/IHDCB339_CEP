%----------------------------------------------------------------------------------------
%	PACKAGES AND OTHER DOCUMENT CONFIGURATIONS
%----------------------------------------------------------------------------------------

\documentclass[11pt]{article}

\bibliography{biblio}

\usepackage{amsmath,amsfonts,stmaryrd,amssymb} % Math packages

\usepackage{enumerate} % Custom item numbers for enumerations

\usepackage[ruled]{algorithm2e} % Algorithms

\usepackage[framemethod=tikz]{mdframed} % Allows defining custom boxed/framed environments

\usepackage{biblatex}

\usepackage{listings} % File listings, with syntax highlighting
\lstset{
	basicstyle=\ttfamily, % Typeset listings in monospace font
}

\usepackage{graphicx}

\usepackage{hyperref}

\makeatletter
\newcommand{\addloflink}[1]{% \addloflink{<URL>}
  \addtocontents{lof}{\begingroup\def\protect\@dotsep{10000}% Remove dots in LoF for this entry
    \protect\contentsline{figlink}{\protect\numberline{}\url{#1}}{}{}%
    \endgroup}% Restore dots in LoF for future entries
}
\newcommand{\l@figlink}{\@dottedtocline{1}{1.5em}{2.3em}}
\makeatother

\begin{document}

%----------------------------------------------------------------------------------------
%	TITLE PAGE
%----------------------------------------------------------------------------------------

\begin{titlepage} % Suppresses displaying the page number on the title page and the subsequent page counts as page 1
	\newcommand{\HRule}{\rule{\linewidth}{0.5mm}} % Defines a new command for horizontal lines, change thickness here
	
	\center % Centre everything on the page
	
	%------------------------------------------------
	%	Headings
	%------------------------------------------------
	
	\textsc{\LARGE University of Namur}\\[1.5cm] % Main heading such as the name of your university/college
	
	\textsc{\Large State of the art}\\[0.5cm] % Major heading such as course name
	
	\textsc{\large IHDCB339}\\[0.5cm] % Minor heading such as course title
	
	%------------------------------------------------
	%	Title
	%------------------------------------------------
	
	\HRule\\[0.4cm]
	
	{\huge\bfseries Complex Event Processing for Internet of Things}\\[0.4cm] % Title of your document
	
	\HRule\\[1.5cm]
	
	%------------------------------------------------
	%	Author(s)
	%------------------------------------------------
	
	\begin{minipage}{0.4\textwidth}
		\begin{flushleft}
			\large
			\textit{Author}\\
			Kenny \textsc{Warszawski} % Your name
		\end{flushleft}
	\end{minipage}
	~
	\begin{minipage}{0.4\textwidth}
		\begin{flushright}
			\large
			\textit{Supervisor}\\
			Moussa \textsc{Amrani} % Supervisor's name
		\end{flushright}
	\end{minipage}
	
	%------------------------------------------------
	%	Date
	%------------------------------------------------
	
	\vfill\vfill\vfill % Position the date 3/4 down the remaining page
	
	{\large\today} % Date, change the \today to a set date if you want to be precise
	
	%------------------------------------------------
	%	Logo
	%------------------------------------------------
	
	\vfill\vfill
	\includegraphics[width=0.2\textwidth]{placeholder.png}\\[1cm] % Include a department/university logo - this will require the graphicx package
	 
	%----------------------------------------------------------------------------------------
	
	\vfill % Push the date up 1/4 of the remaining page
	
\end{titlepage}

%----------------------------------------------------------------------------------------

%----------------------------------------------------------------------------------------
%	Beginning
%----------------------------------------------------------------------------------------

\section{Introduction}

The Internet of Things is omnipresent and the amount of connected devices is increasing day by day. The Internet of Things is different from a classic information system by its capability for integrating with the physical world. Those devices are used in plenty of sectors (health, industry, domotic systems, ...) and their users can be both professionals and individuals. Nowadays, we can see the apparition of connected devices for home to facilitate our daily life. (e.g: Google Home, Nest, Philips Hue) IoT is also an interesting technology for Smart Cities use case. We can imagine a city where traffic lights are optimised with the city mobility to avoid traffic jams for example.
However, an important problem arises in this type of architecture. How is it possible to handle such an important data traffic efficiently ? Indeed, if an entire city has a huge amount of connected objects, the data flow to be processed is massive. Therefore efficient data processing mechanism needs to be put in place to handle such a flow.

%--------------
% \begin{itemize}
%    \item \textbf{Item 1:} My item \cite{1}
% \end{itemize}
%----------------

%----------------------------------------------------------------------------------------
%	CONCEPTS
%----------------------------------------------------------------------------------------

\section{Data Treatment Complexity}


(Penser à parler au fog et cloud computing qui a beaucoup paru dans les articles)
Attention ne pas parler uniquement du CEP mais également d'autres techniques.

\subsection{Technique A}

\subsection{Technique B}

\subsection{Complex Event Processing}

%----------------------------------------------------------------------------------------


\section{Chap 2}

\subsection{Sub-Section}


\section{Chap 3}

\subsection{Sub-section 1}

\subsection{Sub-section 2}

\subsubsection{Sub-sub-section 1}

\subsubsection{Sub-sub-section 2}

\section{Methodology}




%----------------------------------------------------------------------------------------

\section{Conclusions}

\listoffigures

\printbibliography

\end{document}
